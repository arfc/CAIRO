\begin{abstract}
Calls for decarbonization led to rapid growth of solar photovoltaic cells and
wind turbines. These energy sources vary based on weather
features such as solar irradiance and wind speed. This variability challenges
grid operators to reliably meet demand through scheduling dispatchable
resources. Weather and energy data
from the diverse microgrid at the University of Illinois at Urbana-Champaign are
used to develop accurate forecasts for total electricity demand, wind power, and
solar power with echo state networks. The influence of various weather
parameters on forecast accuracy are evaluated. Simulation results show that
forecasts can be significantly improved by some additional predictors. These
improvements are comparable to the accuracy of more sophisticated algorithms.
\end{abstract}
