\begin{abstract}
Solar photovoltaic cells and wind turbines are two of the fastest growing
forms of renewable energy worldwide. These sources of energy are highly variable
due to their reliance on weather features such as solar irradiance and wind
speed. This variability challenges the ability for grid operators to reliably
meet demand through scheduling dispatchable resources. Weather and energy data
from the diverse microgrid at the University of Illinois at Urbana-Champaign is
used to develop accurate forecasts for total electricity demand, wind power, and
solar power with echo state networks. The influence of different weather
parameters on forecast accuracy are evaluated. Simulation results show that
forecasts can be significantly improved by some additional predictors. These
improvements are comparable to the accuracy of more sophisticated algorithms.
\end{abstract}
