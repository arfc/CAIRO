\section{Discussion}

The forecast accuracy of our \gls{esn} for the Lorenz model does not persist
for quite as long as in other work \cite{pathak_using_2017}. However, our
model successfully replicates the environment that produces the Lorenz
Attractor. Further, each randomly instantiated reservoir has a unique set of
optimal hyper-parameters. It is impossible to replicate the exact conditions of
other works without information about a seed for the random state. We have
included this information for future work to compare with our results.

For each target variable (demand, wind, and solar) we found that sun
elevation angle, while not always the best, improved the forecast error in
every case. We hypothesize that additional weather features effect model
performance due to their temporal
complexity relative to the target variable. Electricity demand,
for example, is quite ``predictable," and therefore has low complexity.
Meanwhile, air
temperature and other weather related variables are less predictable. Thus,
adding air temperature as a model input increases the total system complexity
and weakens performance. Further, solar elevation angle is completely
deterministic, perfectly predictable, and either improved,
or neutrally effected, model performance in every case. Like electricity
demand, solar elevation angle has both diurnal and annual periodicity but has
lower complexity than air temperature and electricity demand itself.
Conversely, solar and wind energy are both nonlinear functions of many weather
variables and consequently have greater complexity than air temperature. This
means that adding a temperature feature as a model input will likely decrease
the total system complexity and improve the renewable energy forecasts. Including wind
speed only improved the wind energy forecasts, likely because it has greater
complexity than solar energy but less than wind energy. Relative humidity has
an inconsistent and poorly understood effect on model performance. It improved
the forecast for 48-hour ahead electricity demand but worsened it for the 4-hour
ahead forecast, as shown in Table \ref{tab:demand48} and Table
\ref{tab:demand04}. The opposite trend occurred for solar energy.
Quantifying the predictability and complexity of these systems is in progress.
\textit{Weighted permutation entropy} is a good measure for this type of
complexity \cite{fadlallah_weighted-permutation_2013,garland_model-free_2014, pennekamp_intrinsic_2019}.

These results point to an important disadvantage of using \glspl{esn} to
forecast renewable energy: Although simple and fast, \glspl{esn} remain a black box. We assume that there exists some underlying dynamics that
can be ``learned,'' but we cannot observe the learning process nor extract
important features from \glspl{esn}.

We decided the forecast lengths based on the requirements for improved
economics and planning mentioned in the literature
 \cite{wang_quantifying_2016,mc_garrigle_quantifying_2015,brancucci_martinez-anido_value_2016}. The \gls{esn} model performed reasonably well at predicting
 4-hours ahead but did not improve on the state-of-the-art
 \cite{wang_quantifying_2016,powers_weather_2017}. The model did not perform
 well at the 48-hour ahead forecasts, potentially due to the lack of higher
 resolution data. \glspl{esn} are able to predict highly non-linear
  systems \cite{jaeger_harnessing_2004,lukosevicius_reservoir_2009}, yet
 using hourly data could add spurious complexity that confounds the model
 \cite{garland_model-free_2014}

\subsection{Future Work}
One appealing avenue of continued work is to leverage \glspl{esn} to generate
synthetic data that respects real dynamics. Sythetic data are often useful for
other machine learning or optimization algorithms. Typically, these data are
produced by sampling from an Auto-Regressive Moving Average (ARMA) model \cite{baker_optimal_2018, garcia_dynamic_2016}, which tacitly assumes the
training data can be made stationary. \glspl{esn} can replicate
the environment of a dynamical system, although it remains unclear how far
in the future this behavior persists \cite{pathak_using_2017,pathak_model-free_2018}.
Future work will also explore the effect of data resolution on model
performance, as well as evaluate improvements to the \gls{esn}
algorithm.
