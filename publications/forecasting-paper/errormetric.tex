\subsection{Performance Metrics}
We measure the accuracy of the model using two error metrics:
mean absolute error (MAE) and root mean squared error (RMSE).
These are defined as
\begin{align}
  \text{MAE} &= \frac{1}{N}\sum_{i=1}^N\left|y_i - \hat{y}_i\right|\\
  \text{RMSE} &= \sqrt{\frac{1}{N}\sum_{i=1}^N\left(y_i - \hat{y}_i\right)^2}
\end{align}
The MAE measures the expected error throughout the forecast horizon. The RMSE
indicates the presence of large but infrequent errors. Since the data were
normalized by system capacity \cite{wang_quantifying_2016}, the error metrics
are easily interpretable.
In order to compare how each individual weather input either improved or
worsened the forecast we calculated a ``percent improvement'' over the
univariate case (i.e. a demand prediction based only on historical demand data).
This percent improvement is calculated by
\begin{align}
  \text{\% Improvement} &= \frac{\hat{e} - e}{e}\times 100, \text{ [-]}
\end{align}
where $e$ is the univariate error and $\hat{e}$ is the duovariate error. The
sign indicates the direction of change in error.
