\section{Introduction}
\subsection{Motivation}
In response to the rising threat of climate change, many countries have
prioritized reducing carbon emissions. The 2015 Paris
Agreement aims to prevent the global temperature from rising more than 1.5
$^\circ$C above pre-industrial levels \cite{noauthor_paris_nodate}. Virtually
all current plans to reduce carbon emissions depend on increasing the share of
energy production by renewable and clean energy sources, especially solar and
wind \cite{cany_nuclear_2018, chilvers_realising_2017,99th_general_assembly_illinois_2016,isee_illinois_2015}.
 While solar and wind generate zero carbon emissions, these
forms of electricity generation increase grid variability, which can lead to
blackouts and power system failures
\cite{haes_alhelou_survey_2019}. Further, even modest variable energy
penetration negatively affects the economics of other clean energy sources,
such as nuclear power
\cite{cany_nuclear_2018,keppler_carbon_2011,illinois_commerce_commision_icc_potential_2015}. This may force nuclear plants to shut down prematurely, at the
precise moment that all clean sources of energy are most needed. Some existing
work quantified the economic benefit of improving
renewable energy forecasts \cite{wang_quantifying_2016, mc_garrigle_quantifying_2015, brancucci_martinez-anido_value_2016}.
Improving renewable energy forecasts can
mitigate some of the negative side effects of variability. The economic
benefits of better  forecasts include: reduced costs compared to building
storage devices \cite{wang_quantifying_2016}; curtailment reduction and more
efficient use of non-renewable sources \cite{mc_garrigle_quantifying_2015}; and
modest load-following from nuclear and biomass
generators, which are unable to follow rapid changes in demand
\cite{brancucci_martinez-anido_value_2016}. Most proposed forecasting
improvements involve new algorithms or machine learning techniques. However,
one of the simplest approaches to improving forecasts is to improve the
training data for such algorithms. There is a veritable zoo of weather
parameters that can supplement target training data, but we don't know \textit{a
priori} which of these parameters will be helpful or detrimental to model
performance. In this paper, we evaluate several common parameters for use
in renewable energy forecasting with \glspl{esn}.

\subsection{Why Echo State Networks}
\glspl{esn} have several appealing features. Simplicity, consisting only
of a large, sparse, reservoir and a single output layer
\cite{lukosevicius_practical_2012}; flexibility and generalizability,
while other network architectures require significant fine tuning
\cite{liu_deterministic_2019}; and speed, due to their simple structure and
few trainable weights relative to other neural networks. The \gls{esn} network
architecture eliminates the need for complicated data pre-processing, such as
feature extraction, that is required for other machine learning  and statistical
algorithms \cite{lazos_optimisation_2014, chen_day-ahead_2017}.
\glspl{esn} can also outperform other prediction techniques
\cite{jayawardene_comparison_2014,jayawardene_comparison_2015,shi_energy_2016,chitsazan_wind_2019, hu_forecasting_2020}.

Classical \glspl{esn} have previously been used to forecast demand, wind
energy, and solar energy
\cite{deihimi_application_2012,jayawardene_comparison_2015,hu_forecasting_2020}.
Typically, \glspl{esn} make extreme short term predictions, on the
order of seconds or minutes
\cite{chen_novel_2019,wang_echo_2019,chitsazan_wind_2019}, one-hour ahead
\cite{shi_energy_2016}, and up to one ahead
\cite{deihimi_application_2012}. Forecasts must be at least 4-hours ahead
\cite{wang_quantifying_2016} to 48-hours ahead
\cite{wang_quantifying_2016,mc_garrigle_quantifying_2015,brancucci_martinez-anido_value_2016}
to aid resource scheduling and grid-scale energy economy. In this work we use
a classic \gls{esn} architecture to forecast total demand, wind production, and
solar production, 4-hours and 48-hours ahead.

Approaches in the literature to improve the forecasting capability of
the basic \gls{esn} include: adding multiple reservoirs
\cite{hu_forecasting_2020,gallicchio_deep_2019,yao_novel_2019,li_multi-reservoir_2020}, including non-linear units \cite{holzmann_echo_2008, chitsazan_wind_2019}, combining with other network architecture
\cite{chen_novel_2019, lopez_wind_2018}; and using a particle swarm approach
\cite{chouikhi_pso-based_2017,wang_echo_2019}. Some work indicates that
including weather parameters may be useful for renewable energy forecasting
\cite{li_echo_2019,chitsazan_wind_2019}, but none have demonstrated the effect
each parameter has on model performance. The primary goal of this work
is to fill that gap.

\subsection{Contributions}
In this work, we use \glspl{esn} to predict three key values: total
electricity demand, wind energy production, and solar energy production. We
split these tasks into further sub tasks, predicting 4-hours ahead and 48-hours
ahead. These predictions facilitate scheduling and grid planning
because current market rules put renewable energy on the grid first, forcing
conventional power generators to work around this variability
\cite{wang_quantifying_2016}. Using \glspl{esn} to make predictions two days
ahead is unique to this paper since the longest predictions by \glspl{esn} in
the literature only reach one day ahead \cite{deihimi_application_2012}.
Finally, we repeat these tasks with several commonly used weather parameters
and evaluate their effect on model performance. The need to consider exogenous
meteorological inputs has been noted in the literature
\cite{chitsazan_wind_2019}. Suprisingly, sun elevation is
seldom used as a correlated quantity for energy demand and wind power.

In Section 2 of this paper, we discuss how data
were selected and processed, and we review \glspl{esn}. Section 3 shows a
benchmarking
exercise for our \gls{esn} implementation and presents the results. We discuss
the results and future implications in Section 4.
