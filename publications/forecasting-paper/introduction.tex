\section{Introduction}
\subsection{Motivation}
Reducing carbon emissions has become a priority for many countries in response
to the rising threat of climate change. The goal set by the 2015 Paris
Agreement is to prevent the global temperature from rising more than 1.5
$^\circ$C above pre-industrial levels \cite{noauthor_paris_nodate}. Virtually
all current plans to reduce carbon emissions depend on increasing the share of
energy production by renewable and clean energy sources, especially solar and
wind energy \cite{cany_nuclear_2018, chilvers_realising_2017,99th_general_assembly_illinois_2016,isee_illinois_2015}.
 While solar and wind are low-carbon sources, these
forms of electricity generation are variable and unpredictable. This variability
is found to be major cause of blackouts and power system failures
\cite{haes_alhelou_survey_2019}. Further, even modest penetrations of renewable
energy negatively affect the economics of other types of clean energy, such as
nuclear power
\cite{cany_nuclear_2018,keppler_carbon_2011,illinois_commerce_commision_icc_potential_2015}. This may force nuclear plants to shutdown prematurely, at the
precise moment clean sources of energy are most needed. There has been
some work done to quantify the economic benefit of improving forecasts of
renewable energy \cite{wang_quantifying_2016, mc_garrigle_quantifying_2015, brancucci_martinez-anido_value_2016}. Some of the benefits of improving
forecasts are: 1) It is often cheaper than building storage devices
\cite{wang_quantifying_2016}. 2) Would reduce curtailment and allow for
efficient use of non-renewable sources \cite{mc_garrigle_quantifying_2015}.
3) Enable a slight, but important, amount of load-following from nuclear and
bio-mass generators which are not designed for rapid load following
\cite{brancucci_martinez-anido_value_2016}. Most proposed forecasting
improvements involve new algorithms or machine learning techniques. However,
one of the simplest approaches to improving forecasts is to improve the
training data for such algorithms. There is a veritable zoo of weather
parameters that can supplement target training data and we don't know \textit{a
priori} which of these parameters will be helpful or detrimental to model
performance. In this paper, we evaluate several of these parameters for use
in renewable energy forecasting with \glspl{esn}.
\subsection{Why Echo State Networks}
\glspl{esn} have several appealing features. They are simple, consisting only
of a large sparse reservoir and a single output layer
\cite{lukosevicius_practical_2012}; flexible, \glspl{esn} are generalizable
where other network architectures require significant fine tuning
\cite{noauthor_deterministic_2019}; fast, due to their simple structure and very few trainable weights relative to other neural networks.

Classical \glspl{esn} have previously been used to forecast demand, wind energy, and solar energy \cite{deihimi_application_2012,}.
\subsection{Review of Forecasting Techniques}
