\subsection{Data Selection and Processing}

The \gls{uiuc} Solar Farm 1.0 dashboard provides data for the solar energy
generated  on campus \cite{alsoenergy_university_2019}. The \gls{uiuc}
Facilities and Services Department shared proprietary data for campus
electricity demand and wind energy with us \cite{marquissee_campus_2019}. All
data had hourly resolution. Weather data were retrieved from the
\gls{noaa}\cite{national_center_for_environmental_information_find_nodate} for
two locations: Champaign, IL, where \gls{uiuc} is located, and Lincoln, IL,
where Railsplitter Windfarm is located. \gls{uiuc} has a power purchase
agreement with Railsplitter Windfarm \cite{breitweiser_wind_2016}.

In the case of \gls{uiuc} solar data, significant portions were missing due to
instrument failure. In order to fill in this missing data, we calculated the
theoretical solar energy production based on irradiance data from OpenEI
\cite{noauthor_national_nodate, garcia_nuclear_2015} with
\begin{align}
  P &= G_T\eta_{ref}\tau_{pv}A\left(1-\gamma\left(T-25\right)\right) \text{ } \left[W\right]\\
  \intertext{where}
  P &= \text{ the total power of the solar farm}\nonumber\\
  G_T &= \text{ the total incident solar irradiance}\nonumber\\
  &= P_{DNI}\cos\left(\beta+\delta-lat\right)+ P_{DHI}\left(\frac{180-\beta}{180}\right) \left[\frac{W}{m^2}\right]\\
  \intertext{and}
  \delta &= \text{ the solar declination angle}\nonumber\\
  &= 23.44\sin\left(\left(\frac{\pi}{180}\right)\left(\frac{360}{365}\right)(N+284)
  \right) \left[^\circ \right]\\
  \eta_{ref} &= \text{ conversion efficiency [-]}\nonumber\\
  \tau_{pv} &= \text{ transmittance [-]}\nonumber\\
  \gamma &= \text{ thermal coefficient [-]}\nonumber\\
  A &= \text{ solar panel coverage  $[m^2]$}\nonumber\\
  P_{DNI}&= \text{direct normal irradiance  $\left[\frac{W}{m^2}\right]$}\nonumber\\
  P_{DHI}&= \text{diffuse horizontal irradiance  $\left[\frac{W}{m^2}\right]$}\nonumber\\
  \beta &= \text{tilt angle of the solar panels [$^\circ$]}\nonumber\\\nonumber
\end{align}
We also calculated the solar elevation angle, $\alpha$, using
coordinates for the \gls{uiuc} Solar Farm 1.0
\cite{us_department_of_commerce_esrl_nodate, meeus_astronomical_1998},
\begin{align}
  \alpha &= \sin^{-1}\left[\sin(\delta)\sin(\phi)+\cos(\delta)\cos(\phi)\cos(\omega)\right] \left[^\circ \right]
  \intertext{where}\nonumber
  \delta &= \text{declination angle  [$^\circ$]}\\\nonumber
  \phi &= \text{latitude of interest  [$^\circ$]}\\\nonumber
  \omega &= \text{hour angle  [$^\circ$]}\\\nonumber
\end{align}
Finally, we normalized all of the data using the infinity norm
\begin{align}
  \norm{\mathbf{x}}_\infty \equiv \text{max}\left|x_i\right|.
\end{align}
The infinity norm is equivalent to normalizing by the system capacity. This
simplifies the comparison of our results between
tasks whose training data have vastly different magnitudes. This normalization
also makes it possible to compare results with other work and is consistent
with the recommendation from Kobylinski et al. (2020) \cite{kobylinski_high-resolution_2020}. Table \ref{tab:capacity} gives the maximum value for each
system.

\begin{table}[h]
  \centering
  \caption{Description of the size of the \gls{uiuc} microgrid}
  \label{tab:capacity}
  \begin{tabular}{l r}
    \hline
    System & Maximum Value\\
    \hline
    Electricity Demand & 81.6 [MW]\\
    Solar Energy & 4.7 [MW]\\
    Wind Energy & 8.8 [MW]\\
    \hline
  \end{tabular}
\end{table}
