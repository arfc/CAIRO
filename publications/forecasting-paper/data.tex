\subsection{Data Selection and Processing}

All data predicting demand, wind energy, and solar energy on the \gls{uiuc}
campus are from the \gls{uiuc} Solar Farm 1.0 dashboard
\cite{alsoenergy_university_2019} and proprietary data shared with us courtesy
of the \gls{uiuc} Facilities and Services Department. All data had hourly
resolution. Weather data were retrieved from the
\gls{noaa}\cite{national_center_for_environmental_information_find_nodate} for
two locations: Champaign, IL, where \gls{uiuc} is located, and Lincoln, IL,
where Railsplitter Windfarm is located. \gls{uiuc} has a power purchase
agreement with Railsplitter Windfarm \cite{breitweiser_wind_2016}.

In the case of \gls{uiuc} solar data, significant portions were missing due to
instrument failure. In order to fill in this missing data, we calculated the
theoretical solar energy production based on irradiance data from OpenEI
\cite{noauthor_national_nodate}. The solar output is given by
\cite{garcia_nuclear_2015}
\begin{align}
  P &= G_T\eta_{ref}\tau_{pv}A\left[1-\gamma\left(T-25\right)\right] \text{ } \left[W\right]\\
  \intertext{where}
  G_T &= P_{DNI}*\cos\left(\beta+\delta-lat\right)\\
  &+ P_{DHI}*\left(\frac{180-\beta}{180}\right) \left[\frac{W}{m^2}\right]\nonumber\\
  \intertext{where}
  \delta =&
  23.44\sin\left(\left(\frac{\pi}{180}\right)\left(\frac{360}{365}\right)(N+284)
  \right) \left[\text{degrees} \right]\\
  &\eta, \tau, \gamma \text{ are solar panel properties}\nonumber\\
  P_{DNI}& \text{ is the direct normal irradiance}\nonumber\\
  P_{DHI}& \text{ is the diffuse horizontal irradiance}\nonumber\\
  \beta &\text{ is the tilt angle of the solar panels}\nonumber\\\nonumber
\end{align}
The solar elevation angle, $\alpha$, was also calculated
\cite{us_department_of_commerce_esrl_nodate, meeus_astronomical_1998} using
coordinates for the \gls{uiuc} Solar Farm 1.0.
\begin{align}
  \alpha &= \sin^{-1}\left[\sin(\delta)\sin(\phi)+\cos(\delta)\cos(\phi)\cos(\omega)\right]
  \intertext{where}\nonumber
  \delta &\text{ is the declination angle}\\\nonumber
  \phi &\text{ is the latitude of interest}\\\nonumber
  \omega &\text{ is the hour angle}\\\nonumber
\end{align}
Finally, we normalized all of the data using the infinity norm
\begin{align}
  \norm{\mathbf{x}}_\infty \equiv \text{max}\left|x_i\right|.
\end{align}
The infinity norm is equivalent to normalizing by the system capacity. This is
useful because it simplifies the comparison of our results between
tasks whose training data have vastly different magnitudes. This normalization
also makes it possible to compare results with other works and is consistent
with the recommendation from Kobylinski et al. (2020) \cite{kobylinski_high-resolution_2020}. Table \ref{tab:capacity} gives the maximum value for each
system.

\begin{table}[h]
  \centering
  \caption{Description of the size of the \gls{uiuc} microgrid}
  \label{tab:capacity}
  \begin{tabular}{c|r}
    \hline
    System & Maximum Value\\
    \hline
    Electricity Load & 81.6 [MW]\\
    Solar Energy & 4.7 [MW]\\
    Wind Energy & 8.8 [MW]
  \end{tabular}
\end{table}
